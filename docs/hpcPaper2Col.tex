\documentclass[10pt,psfig,letterpaper,twocolumn]{article}
\usepackage{geometry}
\usepackage{graphics}
\usepackage{setspace}
\usepackage{natbib}
%\usepackage{named}
%\usepackage[super,sort]{natbib}
%\usepackage[numbers,sort&compress]{natbib}

%\renewcommand{\rmdefault}{ptm}
%\renewcommand{\bfdefault}{b}
%\usepackage{helvet}
\usepackage[scaled=0.9]{helvet}
%\usepackage{courier}
%\normalfont % in case the EC fonts aren't available
%\usepackage[T1]{fontenc}

%\usepackage{multicol}
%\usepackage{doc}
%\usepackage{float}

%\newcommand\bibname{\small{REFERENCES}}
\singlespacing
\paperwidth 8.5in
\paperheight 11in
\oddsidemargin 0in
%\oddsidemargin -0.25in
%\topskip 0in
%\topsep 0in
\headsep 1.3cm 
%\headsep 6mm 
%\headheight 0in 
%\topmargin 0in
%\topmargin -0.25in
%\topmargin -0.85in
\geometry{left=0.75in,top=0.75in,right=0.75in,bottom=1in}
%\evensidemargin 
\textwidth 7in 
\textheight 9.25in
\columnsep 0.4in
%\headheight 0cm
%\footheight 1cm
\footskip 0in 
%\partopsep -0.5cm
%\textfloatsep 0.3cm
%\intextsep 0.3cm

%\makeatletter
%\newenvironment{tablehere}{\def\@captype{table*}}{}

%\newenvironment{figurehere}
%  {\def\@captype{figure}}
%  {}
%\makeatother
\renewcommand{\bibname}{REFERENCE}
\pagestyle{empty} 
\begin{document}
\bibliographystyle{acm} 
%\bibliographystyle{named} 
%\include{ref}


%\pagestyle{plain} 
%\def\Try#1#2{{\fontfamily{#1}\selectfont}}
\title{\fontfamily{phv}\selectfont{\huge{\bfseries{Crowd-sourcing Invasive Alien Species data}}}}
\author{
{\fontfamily{ptm}\selectfont{\large{\bfseries{Stephen Pike}}}}\thanks{... email: steve@synfinity.net }, \and
{\fontfamily{ptm}\selectfont{\large{\bfseries{Daniel Jones}}}}\thanks{Swansea University. email: daniel.ll.jones@gmail.com}, \and 
{\fontfamily{ptm}\selectfont{\large{\bfseries{Martin Allison}}}}\thanks{Potato, Bristol. email: martin.allison@potatolondon.com}, \and
{\fontfamily{ptm}\selectfont{\large{\bfseries{Mark Allison}}}}\thanks{... email: mallison77@gmail.com}\\
}
\date{}
\maketitle
\thispagestyle{empty}
\begin{abstract}
Invasive Alien Species (IAS) are a global problem, causing reductions in biodiversity, and damaging urban and ... environments. The ability to control and manage them is hampered by a lack of spatially framed data, or by the use of temporally poor data. With the ubiquity of the smart phone, and cheap access to data networks, it should be possible to plug the spatial and temporal holes in IAS [[SPREAD]] data, using crowd-sourcing. That is, allowing the general public to log data in the name of `citizen science', providing much-needed, real-time data to scientists whilst also engaging the public. IAS-ESS (iAssess) aims to fulfil this need.
\end{abstract}
{\bf Keywords:}
Public engagement, Invasive Alien Species, crowd-sourcing, management, mobile phones.
\section*{\fontfamily{phv}\selectfont{\normalsize{\bfseries{INTRODUCTION}}}}

The rise of Invasive Alien Species (IAS) is causing degradation to biodiversity, worldwide [[CITE STUFF PLEASE DAN]]. It is fortunate then that in this, the era of ubiquitous mobile computing, citizen science is also on the rise \cite{Silvertown:2009p44145}. Through the use of mobile technology, and with the pervasion of the Internet, it is now possible to quickly gather data from millions of people; the world can be perceived as a network of human sensors \cite{Goodchild:2007}. In \cite{Hart:2006}, Environmental Sensor Networks (ENS) are described as:

\begin{quote}\em{
[comprising of] an array of sensor nodes and a communications system which allows their data to reach a server...
}
\end{quote}

It is clear that one can include the human sensor network in this classification, and indeed, citizen science should be embraced in this manner.

\section*{\fontfamily{phv}\selectfont{\normalsize{\bfseries{}}}}

\subsection*{\fontfamily{phv}\selectfont{\normalsize{\bfseries{Results: 8-block partition}}}}

\subsubsection*{\fontfamily{phv}\selectfont{\normalsize{\it{With one slave processor}}}}

\section*{\fontfamily{phv}\selectfont{\normalsize{\bfseries{Conclusion}}}}

\pagebreak[4]
\vspace*{5.52in}
\bibliography{ref}
\end{document}
\documentclass[10pt,psfig,letterpaper,twocolumn]{article}
\usepackage{geometry}
\usepackage{graphics}
\usepackage{setspace}
\usepackage{natbib}
\usepackage{har2nat}
%\usepackage{named}
%\usepackage[super,sort]{natbib}
%\usepackage[numbers,sort&compress]{natbib}

%\renewcommand{\rmdefault}{ptm}
%\renewcommand{\bfdefault}{b}
%\usepackage{helvet}
\usepackage[scaled=0.9]{helvet}
%\usepackage{courier}
%\normalfont % in case the EC fonts aren't available
%\usepackage[T1]{fontenc}

%\usepackage{multicol}
%\usepackage{doc}
%\usepackage{float}

%\newcommand\bibname{\small{REFERENCES}}
\singlespacing
\paperwidth 8.5in
\paperheight 11in
\oddsidemargin 0in
%\oddsidemargin -0.25in
%\topskip 0in
%\topsep 0in
\headsep 1.3cm 
%\headsep 6mm 
%\headheight 0in 
%\topmargin 0in
%\topmargin -0.25in
%\topmargin -0.85in
\geometry{left=0.75in,top=0.75in,right=0.75in,bottom=1in}
%\evensidemargin 
\textwidth 7in 
\textheight 9.25in
\columnsep 0.4in
%\headheight 0cm
%\footheight 1cm
\footskip 0in 
%\partopsep -0.5cm
%\textfloatsep 0.3cm
%\intextsep 0.3cm

%\makeatletter
%\newenvironment{tablehere}{\def\@captype{table*}}{}

%\newenvironment{figurehere}
%  {\def\@captype{figure}}
%  {}
%\makeatother
\renewcommand{\bibname}{REFERENCE}
\pagestyle{empty} 
\begin{document}
%\bibliographystyle{named} 
%\include{ref}


%\pagestyle{plain} 
%\def\Try#1#2{{\fontfamily{#1}\selectfont}}
\title{\fontfamily{phv}\selectfont{\huge{\bfseries{Crowd-sourcing Invasive Alien Species data}}}}
\author{
{\fontfamily{ptm}\selectfont{\large{\bfseries{Stephen Pike}}}}\thanks{... email: steve@synfinity.net }, \and
{\fontfamily{ptm}\selectfont{\large{\bfseries{Daniel Jones}}}}\thanks{Swansea University. email: 606373@swansea.ac.uk}, \and 
{\fontfamily{ptm}\selectfont{\large{\bfseries{Martin Allison}}}}\thanks{Potato, Bristol. email: martin.allison@potatolondon.com}, \and
{\fontfamily{ptm}\selectfont{\large{\bfseries{Mark Allison}}}}\thanks{... email: mallison77@gmail.com}\\
}
\date{}
\maketitle
\thispagestyle{empty}
\begin{abstract}
Invasive Alien Species (IAS) are a growing global problem, reducing biodiversity and causing significant negative socioeconomic impacts. Management and control is hampered by a lack of accessible data of adequate spatial and temporal resolution. Smart phone technology and cheap access to data networks may enable crowd-sourcing of IAS data to bridge the `data gap'. By engaging the general public in `citizen science', IAS-ESS (iAssess) aims to provide much-needed, real-time data to scientists.
\end{abstract}
{\bf Keywords:}
Citizen science; Crowd-sourcing; Invasive Alien Species (IAS);  Public engagement; Smart phones.
\section*{\fontfamily{phv}\selectfont{\normalsize{\bfseries{INTRODUCTION}}}}

Invasive Alien Species (IAS) are an increasing global problem with multiple negative impacts upon biodiversity and socioeconomic interests \cite{Vila:2011ft}. Ubiquitous mobile computing, including the rise of  citizen science is also on the rise \cite{Silvertown:2009tw}. Through the use of mobile technology, and with the pervasion of the Internet, it is now possible to quickly gather data from millions of people (known as `crowd-sourcing' \cite{Wired:2011uj}); the world can be perceived as a network of human sensors \cite{Goodchild:2007vt}. In \citet{Hart:2006uz}, Environmental Sensor Networks (ENS) are described as:

\begin{quote}\em{
[comprising of] an array of sensor nodes and a communications system which allows their data to reach a server...
}
\end{quote}

It is clear that one can include the human sensor network in this classification, and indeed, citizen science should be embraced in this manner.


\section*{\fontfamily{phv}\selectfont{\normalsize{\bfseries{Objectives}}}}

\emph{Engage public
Enable scientists to gather data easily
Allow control/management staff to quickly establish where it is needed}

\section*{\fontfamily{phv}\selectfont{\normalsize{\bfseries{Public engagement \& Crowdsourcing}}}}

\emph{Crucial to enabling scientists and management to do their job - lots of eyes on the ground, in an era of `citizen science' and ubiquity of mobile devices.}

\cite{Wightman:2010p44185} analysed various crowd-sourcing websites and crucially, find that those who contribute to a crowd-source task are motivated more by the task itself than any reward. Also, they find that the simpler the task is to perform, the more likely it is to be performed even if it is of low importance to the user. Although perhaps obvious, these findings remain crucial to engaging the public for scientifically important endeavours such as the monitoring of IAS. Thus, iAssess requires minimal setup and privacy concern for the casual user to be able to add their data; there is no sign up process, and the mobile software available is free to download and use.

\emph{Able to quickly capture image and coordinates. }

\emph{Problems of data quality? GPS inaccuracy/mobile location framework? Averaged out by multiple participants.}

Whilst crowd-sourced data gathering by citizen scientists is clearly valuable to academics and the professional scientific community, there remains a problem with data validity and unintended consequences of mis-information. \citet{Schenk:2009ud} note that one cannot control one's crowd, and it is likely to contain both experts and novices (indeed, we posit further that it will contain a high noise to signal ratio, also!) 

In \citet{Somaweera:2010vi}, the authors find that attempts by the public to control the Cane toad (\emph{Bufo marinus}) are often at a cost to the other native anurans, where identification is not sound. \citet{Somaweera:2010vi} find that community groups and awareness programs reduce mis-identification. In iAssess a `validation' system is used to show only those sightings that have been verified by an expert, and these can later be used to feed back to users wishing to perform their own identification. Indeed, a gallery of each taxon is available, with an interactive map -- a sighting made in a particular location where others have already been validated should invoke confidence and persistence in users. With the focus being firmly on sighting and logging, with data for use by management experts, the propensity toward taking personal action against IAS should be reduced. 

 Urge users not to take action themselves. Feedback of `valid' sightings and the gallery of validated sightings. Reinforce correct identification.
Is lots of low quality data better than no data?

While \citet{GarciaLlorente:vx} note that people are more willing to pay for removal rather than prevention of IAS, \citet{Sharp:2011eh} find that removal of IAS can be viewed unfavourably by the public. This is particularly the case where removal is through toxic or destructive processes, and that public engagement is crucial to ameliorating the resistance to this activity. \citet{GarciaLlorente:vx} also found that individuals' opinions of the practice are influenced by their knowldege and understanding of the IAS and its affects. By encouraging the public to partake in the sighting and reporting of IAS, it is hoped that they will feel more comfortable during the subsequent removal/management of the species; particularly if they feel like they have contributed positively. 

Aside from the perceptions of the public on the management of IAS, there has been argument \cite{Brown:2004uj} for dispassionate study of IAS, followed by counter-argument \cite{Larson:2007vs} in favour of an engagement between science and society. The scope of the iAssess project encapsulates some form of public engagement in this context -- dispassionately empowering the public to take hold of the problem, however it is perceived by them, in a form that is ultimately useful to scientists and informative for the public.

\section*{\fontfamily{phv}\selectfont{\normalsize{\bfseries{Research}}}}

\emph{Scientists attempting to establish the effects of IAS need technology to assist - can survey large areas to locate certain species and apply sampling methods (e.g. Jones et al) but not for species like Harmonia, which are fast spreading. Technology has to compete with the speed of spread.}

\section*{\fontfamily{phv}\selectfont{\normalsize{\bfseries{Management}}}}

\emph{In terms of control, management strategies need to consider not only locations but also properties of the species cohort -- size may affect ability to manage quickly, and discovery of different levels of invasion allow for better management plan....
}

\section*{\fontfamily{phv}\selectfont{\normalsize{\bfseries{Conclusion}}}}

\pagebreak[4]
\vspace*{5.52in}
\bibliographystyle{agsm}
\bibliography{papers}
\end{document}

\documentclass[12pt]{article}
\usepackage{hyperref}
\usepackage{listings}
\usepackage{graphicx, multicol}
\usepackage{natbib}
\usepackage{har2nat}
\usepackage{booktabs}
\usepackage{topcapt}
%\pagestyle{empty}
\pagestyle{myheadings}
\markboth{{\sc Pike, Jones, Allison & Allison}}{Crowd-sourcing Invasive Alien Species data}


\begin{document}
\author{Stephen Pike, Daniel Jones, Martin Allison, Mark Allison}
\title{Crowd-sourcing Invasive Alien Species data}

\maketitle
\begin{abstract}


\end{abstract}

\clearpage

\tableofcontents
\listoffigures
\clearpage

\section{Introduction}

Objectives:

Engage public

Enable scientists to gather data easily

Allow control/management staff to quickly establish where it is needed

\section{Public engagement}
Crucial to enabling scientists and management to do their job - lots of eyes on the ground, in an era of `citizen science' and ubiquity of mobile devices.

Able to quickly capture image and coordinates. 

Problems of data quality? GPS inaccuracy/mobile location framework? Averaged out by multiple participants.
problems of mistakes (frogs under friendly fire). Urge users not to take action themselves. Feedback of `valid' sightings and the gallery of validated sightings. Reinforce correct identification.

Is lots of low quality data better than no data?

\section{Research}

Scientists attempting to establish the effects of IAS need technology to assist - can survey large areas to locate certain species and apply sampling methods (e.g. Jones et al) but not for species like Harmonia, which are fast spreading. Technology has to compete with the speed of spread.

\section{Management}

In terms of control, management strategies need to consider not only locations but also properties of the species cohort -- size may affect ability to manage quickly, and discovery of different levels of invasion allow for better management plan....



\bibliographystyle{agsm}
\bibliography{proposal}
\clearpage
\appendix
\begin{center}
    {\bf APPENDIX}
\end{center}

\end{document}